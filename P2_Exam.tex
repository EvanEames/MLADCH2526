\documentclass[11pt]{article} % use larger type; default would be 10pt

\title{P2 -- ML Applications in Digital Cultural Heritage (12009 \& 12017) -- Examination Summary and Grading}
\author{Dr. Anastasia Eleftheriadou \& Dr. Evan Eames}
\date{\today} % Activate to display a given date or no date (if empty),
         % otherwise the current date is printed 

\begin{document}
\maketitle

Students enrolled in both the Seminar and Hands-On (\"Ubung) component of the P2 module are examined together. The goal is for students to attempt to showcase an archaeological / cultural heritage case study by making a short review of the existing literature and doing a hands-on assignment working with messy real-world datasets by employing data cleaning and ML ideation. \\

The procedure is as follows.

\begin{enumerate}
\item Students choose a dataset from a list (https://tinyurl.com/5ddxezjk). There is one dataset per student -- that is, no two students can work on the same dataset.

\item Students conduct a literature review around the dataset. The goal here is to get ideas for how international research teams have used similar data for interesting research purposes (note: it is not necessary that the articles actually employ ML techniques).

\item Students clean and manipulate the dataset. This should be done in a notebook file with each step nicely explained in markdown. The goal is to bring the dataset to a state conducive to further exploration (plotting, ML, etc.) which could be useful for Archaeology.

\item Students write an accompanying paper (3000 - 5000 words). This paper must comprise the following sections:
\begin{enumerate}
\item An introduction to the state-of-the-art relating to the topic of the dataset you've chosen (pottery, lithics, osteoarchaeology, zooarchaeology, archaeobotany, dating, etc.)
\item A summary of the specific dataset you've chosen, as well as a discussion of how it has been cleaned and prepared. This should include a section on difficulties faced in cleaning
\item A discussion of imagined subsequent applications which could now be attempted with your dataset, and how you would attempt them. These should be based on what similar teams have attempted based on your literature review. You should discuss expected challenges and how you would deal with them.
\item \textbf{Note:} Any students who successfully implement ML applications will receive bonus points -- more points for more relevant and ambitious aplications.
\item Finally, a conclusion summarizing the entirety of the project.
\end{enumerate}
\item Students must submit their papers and notebook files together by 23:59 on February 6$^{th}$. 10\% will be deducted for each day beyond this deadline.
\item In the final 2 weeks of the semester students will be invited to a 30 minute ``mini-defense''. This will consist of Anastasia, Evan, and each student. The structure is:
\begin{enumerate}
\item 5 minutes of presentation of the state-of-the-art
\item 10-15 minutes of overview of what has been done for the project (powerpoints can be included, but are not required).
\item 10-15 minutes of questions
\end{enumerate}
\item Additionally, in the final classes each student will do a quick 5 minute lightning summary of their work

\end{enumerate}

\end{document}
